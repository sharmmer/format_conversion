\documentclass[]{article}
\usepackage{lmodern}
\usepackage{amssymb,amsmath}
\usepackage{ifxetex,ifluatex}
\usepackage{fixltx2e} % provides \textsubscript
\ifnum 0\ifxetex 1\fi\ifluatex 1\fi=0 % if pdftex
  \usepackage[T1]{fontenc}
  \usepackage[utf8]{inputenc}
\else % if luatex or xelatex
  \ifxetex
    \usepackage{mathspec}
  \else
    \usepackage{fontspec}
  \fi
  \defaultfontfeatures{Ligatures=TeX,Scale=MatchLowercase}
\fi
% use upquote if available, for straight quotes in verbatim environments
\IfFileExists{upquote.sty}{\usepackage{upquote}}{}
% use microtype if available
\IfFileExists{microtype.sty}{%
\usepackage[]{microtype}
\UseMicrotypeSet[protrusion]{basicmath} % disable protrusion for tt fonts
}{}
\PassOptionsToPackage{hyphens}{url} % url is loaded by hyperref
\usepackage[unicode=true]{hyperref}
\hypersetup{
            pdfborder={0 0 0},
            breaklinks=true}
\urlstyle{same}  % don't use monospace font for urls
\usepackage{graphicx,grffile}
\makeatletter
\def\maxwidth{\ifdim\Gin@nat@width>\linewidth\linewidth\else\Gin@nat@width\fi}
\def\maxheight{\ifdim\Gin@nat@height>\textheight\textheight\else\Gin@nat@height\fi}
\makeatother
% Scale images if necessary, so that they will not overflow the page
% margins by default, and it is still possible to overwrite the defaults
% using explicit options in \includegraphics[width, height, ...]{}
\setkeys{Gin}{width=\maxwidth,height=\maxheight,keepaspectratio}
\IfFileExists{parskip.sty}{%
\usepackage{parskip}
}{% else
\setlength{\parindent}{0pt}
\setlength{\parskip}{6pt plus 2pt minus 1pt}
}
\setlength{\emergencystretch}{3em}  % prevent overfull lines
\providecommand{\tightlist}{%
  \setlength{\itemsep}{0pt}\setlength{\parskip}{0pt}}
\setcounter{secnumdepth}{0}
% Redefines (sub)paragraphs to behave more like sections
\ifx\paragraph\undefined\else
\let\oldparagraph\paragraph
\renewcommand{\paragraph}[1]{\oldparagraph{#1}\mbox{}}
\fi
\ifx\subparagraph\undefined\else
\let\oldsubparagraph\subparagraph
\renewcommand{\subparagraph}[1]{\oldsubparagraph{#1}\mbox{}}
\fi

% set default figure placement to htbp
\makeatletter
\def\fps@figure{htbp}
\makeatother


\date{}

\begin{document}

\subsection{从百度、谷歌搜索指数看三维软件的趋势,Blender(布兰德)、c4d上升,Maya、3dsmax下降,houdini持平}\label{header-n4}

\begin{figure}
\centering
\includegraphics{http://www.bgteach.com/files/default/2017/07-09/023142e6baea872932.jpg}
\caption{}
\end{figure}

很多学员在学习三维软件之前,总会问哪个软件强大,那个软件更厉害?而同学们更多的也只是道听途说而已,今天我们就严谨的拿百度指数来给他家普及一下这些三维动画软件。我们时间设置在了2011年-2017年7月,搜索指数由低到高为:

\begin{enumerate}
\def\labelenumi{\arabic{enumi}.}
\item
  \textbf{houdini}
\item
  \textbf{3dsmax}
\item
  \textbf{blender}
\item
  \textbf{c4d}
\item
  \textbf{maya;}
\end{enumerate}

\begin{itemize}
\item
  增长最猛烈的是c4d,其次是blender
\item
  下降最猛烈的是maya,其次是3dsmax。
\end{itemize}

\begin{figure}
\centering
\includegraphics{http://www.bgteach.com/files/default/2017/07-05/150408872663265553.jpeg}
\caption{}
\end{figure}

\includegraphics{http://www.bgteach.com/files/default/2017/07-05/1504099454ca828573.jpeg}\href{http://www.bgteach.com/files/default/2017/07-09/023142e6baea872932.jpg}{\includegraphics{}}

\begin{itemize}
\item
  \textbf{maya
  }从2015年开始,加入了稳步增长期,直到2016年3月到达巅峰,随后进入了猛烈的下降期。
\item
  \textbf{4cd }从2016年3月开始进入了持续的上升期。
\item
  \textbf{blender }稳定上升,保持每年10\%-20\%左右的上升。
\item
  \textbf{3dsmax} 持续下降,2015年开始低于了blender、c4d。
\item
  \textbf{houdini }依然持平,小幅度的上升。
\end{itemize}

\begin{figure}
\centering
\includegraphics{http://www.bgteach.com/files/default/2017/07-09/021558e1fbef545038.jpeg}
\caption{}
\end{figure}

谷歌搜索指数时间设置为了过去5年来的变化,结果如下:

\begin{itemize}
\item
  \textbf{maya }比较稳定,逐渐的下降。
\item
  \textbf{4cd }2012年最高,目前逐渐下降。
\item
  \textbf{blender }稳定上升,居然处在了榜首。
\item
  \textbf{3dsmax} 高于maya,处于下降趋势。
\item
  \textbf{houdini} 很少的搜索。
\end{itemize}

\begin{figure}
\centering
\includegraphics{http://www.bgteach.com/files/default/2017/07-05/1504099aa65d916975.jpeg}
\caption{}
\end{figure}

\begin{figure}
\centering
\includegraphics{http://www.bgteach.com/files/default/2017/07-05/150410a166fc851295.jpeg}
\caption{}
\end{figure}

\begin{figure}
\centering
\includegraphics{http://www.bgteach.com/files/default/2017/07-05/150410a7a405170746.jpeg}
\caption{}
\end{figure}

\begin{figure}
\centering
\includegraphics{http://www.bgteach.com/files/default/2017/07-05/150410ad6146454154.jpeg}
\caption{}
\end{figure}

\begin{figure}
\centering
\includegraphics{http://www.bgteach.com/files/default/2017/07-05/150411b669b9783338.jpeg}
\caption{}
\end{figure}

从所有软件的需求图谱来看,大部分集中在以下几个模块:

\begin{itemize}
\item
  \textbf{软件下载}
\item
  \textbf{中文软件}
\item
  \textbf{中文教程}
\end{itemize}

\begin{figure}
\centering
\includegraphics{http://www.bgteach.com/files/default/2017/07-05/150412c3986f019577.jpeg}
\caption{}
\end{figure}

\begin{figure}
\centering
\includegraphics{http://www.bgteach.com/files/default/2017/07-05/150412cc4437674317.jpeg}
\caption{}
\end{figure}

\begin{figure}
\centering
\includegraphics{http://www.bgteach.com/files/default/2017/07-05/150413d6151b091994.jpeg}
\caption{}
\end{figure}

\begin{figure}
\centering
\includegraphics{http://www.bgteach.com/files/default/2017/07-05/150442a309d4917932.jpeg}
\caption{}
\end{figure}

\begin{figure}
\centering
\includegraphics{http://www.bgteach.com/files/default/2017/07-05/150442ac2572653321.jpeg}
\caption{}
\end{figure}

从所有软件的地域来看这五个地区活跃度排名前五:

\begin{itemize}
\item
  \textbf{北京}
\item
  \textbf{上海}
\item
  \textbf{广州}
\item
  \textbf{浙江}
\item
  \textbf{江苏}
\end{itemize}

\begin{figure}
\centering
\includegraphics{http://www.bgteach.com/files/default/2017/07-05/150443b336ed139547.jpeg}
\caption{}
\end{figure}

\begin{figure}
\centering
\includegraphics{http://www.bgteach.com/files/default/2017/07-05/150443ba82a5715515.jpeg}
\caption{}
\end{figure}

\begin{figure}
\centering
\includegraphics{http://www.bgteach.com/files/default/2017/07-05/150444c20f05158830.jpeg}
\caption{}
\end{figure}

\begin{figure}
\centering
\includegraphics{http://www.bgteach.com/files/default/2017/07-05/150444c8eb8b368432.jpeg}
\caption{}
\end{figure}

\begin{figure}
\centering
\includegraphics{http://www.bgteach.com/files/default/2017/07-05/150445d10a6b956623.jpeg}
\caption{}
\end{figure}

年龄集中分布在20-40之间:

\begin{itemize}
\item
  \textbf{20-30岁}之间主要为学习阶段,此阶段最多的为c4d,blender最少。
\item
  \textbf{30-40岁}之间多为工作者,或者已经是行业内的高手,从软件来看,blender、maya、houdini的人多于3dsmax、c4d。
\item
  整体来说,此行业男士要明显多于女生。
\end{itemize}

\begin{figure}
\centering
\includegraphics{http://www.bgteach.com/files/default/2017/07-05/150445d5c3d8420108.jpeg}
\caption{}
\end{figure}

\subsection{\texorpdfstring{\textbf{通过搜索指数,我们得出:}}{通过搜索指数,我们得出:}}\label{header-n95}

\begin{itemize}
\item
  MAYA 绝对优势长居榜首,目前猛烈下降;
\item
  C4D 猛烈上升,已经超过了3dsmax;
\item
  Blender 稳步上升,超过了3dsmax;
\item
  3dsmax 稳步下降,低于了blender、c4d、maya;
\item
  houdini 基本持平,无明显提升。
\item
  \textbf{工作者使用的软件为:}maya、blender、houdnin、c4d都有参与。
\item
  \textbf{学习者学习的软件为:}c4d、houdini、maya
\item ~
  \subsubsection{\texorpdfstring{\textbf{为什么Maya下降猛烈?}}{为什么Maya下降猛烈?}}\label{header-n112}
\end{itemize}

\begin{enumerate}
\def\labelenumi{\arabic{enumi}.}
\item
  Maya从2015年开始,加入了稳步增长期,直到2016年3月到达巅峰,这个时间段正好是各大培训班抛弃了3dsmax转战maya培训的直接结果。随后进入了持续的下降期。
\item
  我国实际工作并非全部依赖maya,maya的软件优势很难在工作中发挥出优势。
\item
  由于c4d等软件的强大整合性,易用性非常适合制作小型的视觉、三维动画。
\item
  我国超大型特效公司稀缺,无法发挥团队优势,配合协作无法开展。
\item
  游戏引擎的飞速发展,抢占了一部分maya的功能。
\item
  对于公司运营来说,正版考虑的话,成本过高。
\end{enumerate}

\begin{itemize}
\item ~
  \subsubsection{\texorpdfstring{\textbf{为什么C4D上升猛烈?}}{为什么C4D上升猛烈?}}\label{header-n128}
\end{itemize}

\begin{enumerate}
\def\labelenumi{\arabic{enumi}.}
\item
  c4d从2016年3月开始进入了持续的上升期,依然坚挺着稳步增长。
\item
  大量的培训班摒弃maya、3dsmax,转头拥抱c4d,免费找到的学习资源越来越多。
\item
  非常适合我国的企业,适合小于50人的小团队。
\item
  c4d接地气,本身最早的时候是学习max的,后来超越了3dsmax,综合性越来越强。内置的很多模块都很实用,非常适合我国对视觉广告、产品广告的制作。
\item
  考虑正版的话价格相对低廉。
\end{enumerate}

\begin{itemize}
\item ~
  \subsubsection{\texorpdfstring{\textbf{为什么工作者中很多使用的是Blender?}}{为什么工作者中很多使用的是Blender?}}\label{header-n142}
\end{itemize}

\begin{enumerate}
\def\labelenumi{\arabic{enumi}.}
\item
  实时渲染需求催生的新软件需求,blender正好能满足。
\item
  来自blender基金会开源电影项目的影响。
\item
  对于新趋势下的软件选择,游戏引擎强力发展最新趋势,比如VR,让几乎所有的三维软件只变成了一个建模软件,而blender的建模效率应该是这几个软件中比较高的了。
\item
  bledner+unity开发手机游戏的案例很多,比如:《纪念碑谷》,效仿的人会也来越多。
\item
  使用者多为30-40岁的工作者,国内比较早一批的三维动画先驱,英文能力好,可以很快的通过国外网站了解相关的动态,相关的学习资料。
\item
  官方开发最新技术,比如2.8版本点pbr技术,eevee引擎效果直逼unreal虚幻,相信在不久的将来,blender的用户会越来越多。
\item
  0成本,开源免费。
\end{enumerate}

\begin{itemize}
\item ~
  \subsubsection{\texorpdfstring{\textbf{为什么3dsmax快没有人过问了?}}{为什么3dsmax快没有人过问了?}}\label{header-n160}
\end{itemize}

\begin{enumerate}
\def\labelenumi{\arabic{enumi}.}
\item
  众所周知3dsmax的强项一直是建筑可视化,但实时渲染技术的发展甩了3dsmax一条街。
\item
  来自诸如SketchUp的软件围堵打击,进一步将3dsmax的建模工作稀释。
\item
  实时渲染可视化软件的飞速发展,比如:lumion,极速的渲染对3dsmax的原工作流程造成了致命的打击。
\item
  3dsmax+插件=牛x软件的思维已经被c4d、blender这类软件撕碎。
\item
  游戏引擎的飞速发展,抢占了一部分3dsmax的功能。
\item
  对于公司运营来说,正版考虑的话,成本过高。
\end{enumerate}

\subsection{\texorpdfstring{\textbf{我们的建议:}}{我们的建议:}}\label{header-n175}

\begin{itemize}
\item
  在c4d、blender中选择一款你喜欢的软件。
\item
  想在未来参与电影的制作可以选择houdini,这个才是真正的特效大师。
\item
  想往游戏发展的人可以学习blender+一款游戏引擎,潜力无限。
\item
  maya依然是各个培训班的主场,中国电影的未来就靠大家这种话不要再相信了。
\item
  时刻关注最新的技术更新、最新的趋势,比如现在已经不是实时渲染的趋势了,而是实时设计、实时电影的趋势了。
\item
  \textbf{平面设计
  }将会消失,大趋势会发展为品牌设计,包含三维部分,也就是说平面设计的人才会要求会一部分三维软件。
\item
  \textbf{室内设计
  }将会从cad+效果图的模式转向依托实时渲染技术下的实时设计,全面向互联网+的模式转变。
\item
  \textbf{建筑园林 }将会发力BIM系统,从国家到企业,全面的普及。
\item
  \textbf{游戏制作
  }将会依托游戏引擎(这里推荐大家了解下blender的eevee引擎),手游将会持续发展。
\item
  \textbf{影视制作
  }依然是热门,依然会徘徊在5毛特效的怪圈里。导演有钱找国外制作,没钱还压榨国内团队。国内的特效公司会多起来,比如:base公司这样的特效公司会慢慢多起来。饼干依然看好产品广告,知道钱在哪,也好组建团队创业。
\end{itemize}

\end{document}
